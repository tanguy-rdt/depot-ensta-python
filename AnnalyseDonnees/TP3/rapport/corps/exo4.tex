\clearpage

\section{Recoder l’exercice~4 du TD~: sondages}
\vspace{.2cm}

\noindent
À la veille d’une consultation électorale, nous effectuons un sondage. 


%%%%%
\begin{itemize}[label={},itemindent=-2em,leftmargin=2em]
    \item \textbf{Question~3~:} Dans un échantillon représentatif de 1000~personnes, 500~personnes déclarent vouloir
    voter pour Dupond, 250 pour Durand et 50 pour Duroc. Donner les intervalles de confiance à $95\%$
    et $99\%$ du pourcentage de personnes ayant l’intention de voter Dupond, Durand ou Duroc.
\end{itemize}
\vspace{.2cm}


\noindent
Formules utilisées~:

\begin{equation}
    \hat{p} - z_{\frac{\alpha}{2}} \sqrt{\frac{\hat{p}(1-\hat{p})}{n}} \leq p \leq \hat{p} + z_{\frac{\alpha}{2}} \sqrt{\frac{\hat{p}(1-\hat{p})}{n}}
    \label{eq:ic_grand2}
\end{equation}

\begin{center}
    avec \qquad $z_{\frac{\alpha}{2}}=erf^{-1}(1 - \frac{\alpha}{2})$ \qquad et \qquad $\alpha = 1- \frac{IC\%}{100}$
\end{center}

\vspace{.2cm}

\begin{lstlisting}[style=myPython, caption=Code Python question 3, frame=lines]
n = 1000
n_dupond = 500
n_durand = 250
n_duroc = 50

def intervalle_confiance(n, x, ic ):
    p = x/n
    alpha = 1 - ic / 100
    z = stats.norm.ppf(1 - (alpha / 2), loc=0, scale=1)
    borne_inf = p - z * (np.sqrt(p*(1-p)/n))
    borne_supp = p + z * (np.sqrt(p*(1-p)/n))

    return round(borne_inf, 3), round(borne_supp, 3)

borne_inf_dupond_95, borne_supp_dupond_95 = intervalle_confiance(n, n_dupond, 95)
borne_inf_durand_95, borne_supp_durand_95 = intervalle_confiance(n, n_durand, 95)
borne_inf_duroc_95, borne_supp_duroc_95 = intervalle_confiance(n, n_duroc, 95)
borne_inf_dupond_99, borne_supp_dupond_99 = intervalle_confiance(n, n_dupond, 99)
borne_inf_durand_99, borne_supp_durand_99 = intervalle_confiance(n, n_durand, 99)
borne_inf_duroc_99, borne_supp_duroc_99 = intervalle_confiance(n, n_duroc, 99)

print(" Dupond")
print("\t --> Borne inférieur à 95%:", borne_inf_dupond_95)
print("\t --> Borne supérieur à 95%:", borne_supp_dupond_95)
print("\t\t\t ------------------------")
print("\t --> Borne inférieur à 99%:", borne_inf_dupond_99)
print("\t --> Borne supérieur à 99%:", borne_supp_dupond_99)
print(" Durand")
print("\t --> Borne inférieur à 95%:", borne_inf_durand_95)
print("\t --> Borne supérieur à 95%:", borne_supp_durand_95)
print("\t\t\t ------------------------")
print("\t --> Borne inférieur à 99%:", borne_inf_durand_99)
print("\t --> Borne supérieur à 99%:", borne_supp_durand_99)
print(" Duroc")
print("\t --> Borne inférieur à 95%:", borne_inf_duroc_95)
print("\t --> Borne supérieur à 95%:", borne_supp_duroc_95)
print("\t\t\t ------------------------")
print("\t --> Borne inférieur à 99%:", borne_inf_duroc_99)
print("\t --> Borne supérieur à 99%:", borne_supp_duroc_99, end="\n\n")
\end{lstlisting}

\begin{lstlisting}[style=myLog, caption=Résultat du code, frame=lines]
Dupond
    --> Borne inférieur à 95%: 0.469
    --> Borne supérieur à 95%: 0.531
        ------------------------
    --> Borne inférieur à 99%: 0.459
    --> Borne supérieur à 99%: 0.541
Durand
    --> Borne inférieur à 95%: 0.223
    --> Borne supérieur à 95%: 0.277
        ------------------------
    --> Borne inférieur à 99%: 0.215
    --> Borne supérieur à 99%: 0.285
Duroc
    --> Borne inférieur à 95%: 0.036
    --> Borne supérieur à 95%: 0.064
        ------------------------
    --> Borne inférieur à 99%: 0.032
    --> Borne supérieur à 99%: 0.068
\end{lstlisting}

\vspace{.5cm}


%%%%%
\begin{itemize}[label={},itemindent=-2em,leftmargin=2em]
    \item \textbf{Question~4~:} Nous évaluons le pourcentage de personnes ayant l’intention de voter pour un quatrième
    candidat, Duval, à $17\%$. Combien faut-il interroger de personnes pour obtenir une précision de
    $1\%$ pour l’intervalle de confiance (à $95\%$) de la proportion de personnes ayant l’intention de voter
    Duval~?
\end{itemize}
\vspace{.2cm}

\begin{equation}
    n = \left( \sqrt{\frac{Z_{\alpha/2}}{E}} \right)^{2} \hat{p}(1 - \hat{p}) \qquad \text{avec $E$ = erreur = précision}
    \label{eq:n}
\end{equation}

\vspace{.2cm}

\begin{lstlisting}[style=myPython, caption=Code Python question 4, frame=lines]
ic = 95
alpha = 1 - ic / 100
p = 17/100
err = 1/100
z = stats.norm.ppf(1 - (alpha / 2), loc=0, scale=1)
n = (z/err)**2 * p*(1-p)
n = np.ceil(n)

print("Nombre de personne à intérroger:", n, end='\n\n')
\end{lstlisting}

\begin{lstlisting}[style=myLog, caption=Résultat du code, frame=lines]
Nombre de personne à intérroger: 5421.0
\end{lstlisting}

\vspace{.5cm}
