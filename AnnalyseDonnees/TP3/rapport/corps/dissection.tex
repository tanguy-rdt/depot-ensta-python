\section{Dissection d’une gaussienne}
\vspace{.2cm}

\noindent
Données mesurées : 0.82 0.87 0.77 0.96 0.75 0.83 0.87 0.81

%%%%%
\begin{itemize}[label={},itemindent=-2em,leftmargin=2em]
    \item \textbf{Question~1~:} Considérant que l’échantillon a été engendré par une loi gaussienne, donner un intervalle
    de confiance pour son espérance. On utilisera les fonctions tinv, mean et var ou std/t.ppf, np.mean,
    np.std. Préciser toutes les hypothèses que vous retenez.
\end{itemize}
\vspace{.2cm}

\noindent
Dans cette question, l'échantillon est de longueur~$n=8$. \\
Si un échantillon est de longueur~$n<30$ avec une $\sigma^{2}$ inconnu, il faut utiliser le cas suivant~: 
\textit{Intervalles de confiance de la moyenne d’une distribution normale, $\sigma^{2}$ inconnue} \\

\noindent
Ce qui nous amène à utiliser les formules \ref*{eq:ic_normale_inf} et \ref*{eq:ic_normale_sup} pour déterminer l'intervalle de confiance $[I, u]$. 

\begin{figure}[!h]
    \centering
    \begin{minipage}{.48\linewidth}
        \begin{equation}
            I = \overline{X_{n}} - t_{n-1,\frac{\alpha}{2}} \frac{S_{n-1}}{\sqrt{n}}
            \label{eq:ic_normale_inf}
        \end{equation}
    \end{minipage}\hfill\vline
    \begin{minipage}{.48\linewidth}
        \begin{equation}
            u = \overline{X_{n}} + t_{n-1,\frac{\alpha}{2}} \frac{S_{n-1}}{\sqrt{n}}
            \label{eq:ic_normale_sup}
        \end{equation}
    \end{minipage}
\end{figure}

\noindent
Comme $\sigma^{2}$ est inconnue, on peut l'approcher par $S_{n-1}^{2}$ puis en déduire $S_{n-1}$. \\
On peut également rappeler que $t_{n-1,\frac{\alpha}{2}}$ est le quantile d’ordre $\frac{\alpha}{2}$ d’une loi de Student de $n - 1$ degrés de liberté, déterminable en python avec \textit{scipy.stats.t.ppf()}.

\vspace{.2cm}


\begin{lstlisting}[style=myPython, caption=Code Python question 1, frame=lines]
x = np.array([0.82, 0.87, 0.77, 0.96, 0.75, 0.83, 0.87, 0.81])
xn = np.mean(x)
ecart_type_x = np.std(x, ddof=1)
n = len(x)
ic = 95
alpha = 1 - ic / 100
t = stats.t.ppf(1 - (alpha / 2), n - 1)

I = xn - (t * ecart_type_x) / np.sqrt(n)
u = xn + (t * ecart_type_x) / np.sqrt(n)

I = round(I, 3)
u = round(u, 3)

print("Borne inférieur:", I, "\n", "Borne supérieur:", u)
\end{lstlisting}

\begin{lstlisting}[style=myLog, caption=Résultat du code, frame=lines]
Borne inférieur: 0.779 
Borne supérieur: 0.890
\end{lstlisting}

\noindent
On peut donc conclure qu'il y'a $95\%$ de chance que l'espérance ce trouve dans l'intervalle $[0,779; 0,890]$.

\vspace{.5cm}

\clearpage

%%%%%
\begin{itemize}[label={},itemindent=-2em,leftmargin=2em]
    \item \textbf{Question~2~:} Les données sont maintenant [0.84 0.87 0.89 0.73 0.84 0.81 0.88 0.85 0.89 0.79 0.79 0.90
    0.59 0.75 0.67 0.76 0.86 0.88 0.70 0.75 0.81 0.77 0.83 0.84 0.71 0.78 0.59 0.91 0.74 0.68 0.77 0.66
    0.80 0.74 1.02 0.91 0.55 0.84 0.66 0.77]. Considérant que l’échantillon a été engendré par une
    loi gaussienne, donner un intervalle de confiance pour son espérance. On utilisera les fonctions
    (norminv, mean et var ou std ou norm.ppf, np.mean, np.std). Préciser toutes les hypothèses que
    vous retenez. Quel est dans ce cas, l’intérêt de la loi gaussienne~?
\end{itemize}
\vspace{.2cm}

\noindent
Dans cette question l'échantillon est de longueur~$n=40$. \\
Pour un échantillon de longueur~$n>30$ suivant une distribution normale centrée réduite~$\mathcal{N}(0,\,1)$ avec $\sigma^{2}$ inconnu, il faut utiliser le cas suivant~: 
\textit{Intervalle de confiance de la moyenne d’une distribution dans le cas d’un grand échantillon} \\

\noindent
Ce qui nous amène à utiliser la formule~\ref*{eq:ic_grand1} pour déterminer l'intervalle de confiance.

\begin{equation}
    \hat{p} - z_{\frac{\alpha}{2}} \sqrt{\frac{S_{n-1}}{n}} \leq p \leq \hat{p} + z_{\frac{\alpha}{2}} \sqrt{\frac{S_{n-1}}{n}}
    \label{eq:ic_grand1}
\end{equation}

\begin{center}
    avec \qquad $z_{\frac{\alpha}{2}}=erf^{-1}(1 - \frac{\alpha}{2})$ \qquad et \qquad $\alpha = 1- \frac{IC\%}{100}$
\end{center}

\noindent
Comme $\sigma^{2}$ est inconnue, on peut l'approcher par $S_{n-1}^{2}$ puis en déduire $S_{n-1}$. \\
On peut également rappeler que $z_{\frac{\alpha}{2}}$ peut être déterminé avec la table et en python avec \textit{stats.norm.ppf()}.

\vspace{.2cm}

\begin{lstlisting}[style=myPython, caption=Code Python question 2, frame=lines]
x = np.array([0.84, 0.87, 0.89, 0.73, 0.84, 0.81, 0.88, 0.85, 0.89, 0.79, 0.79, 0.90,
              0.59, 0.75, 0.67, 0.76, 0.86, 0.88, 0.70, 0.75, 0.81, 0.77, 0.83, 0.84, 
              0.71, 0.78, 0.59, 0.91, 0.74, 0.68, 0.77, 0.66, 0.80, 0.74, 1.02, 0.91,
              0.55, 0.84, 0.66, 0.77])

p = np.mean(x)
ecart_type_x = np.std(x, ddof=1)
n = len(x)
ic = 95
alpha = 1 - ic / 100
z = stats.norm.ppf(1 - (alpha / 2), loc=0, scale=1)

borne_inf = p - z * (ecart_type_x/np.sqrt(n))
borne_supp = p + z * (ecart_type_x/np.sqrt(n))

borne_inf = round(borne_inf, 3)
borne_supp = round(borne_supp, 3)

print("Borne inférieur:", borne_inf, "\n", "Borne supérieur:", borne_supp)
\end{lstlisting}

\begin{lstlisting}[style=myLog, caption=Résultat du code, frame=lines]
Borne inférieur: 0.755 
Borne supérieur: 0.816
\end{lstlisting}


\noindent
On peut donc conclure qu'il y'a $95\%$ de chance que l'espérance ce trouve dans l'intervalle $[0,755; 0,816]$

\vspace{.5cm}

