\vspace{1cm}
\section{Participation volontaire ou non ?}
\vspace{.2cm}

\noindent
Le tableau suivant résume les résultats d’une enquête menée à la Faculté d’Ingénierie de l’Université
de Porto, auprès de 129 étudiants de première année. L’objectif de l’enquête était d’évaluer l’attitude
des étudiants de première année envers les "rites d’initiation des étudiants de première année". Une des
questions était : "J’ai participé à l’initiation de mon plein gré". \\
Les réponses ont été notées comme suit :

\begin{enumerate}
    \item Totalement en désaccord
    \item En désaccord
    \item Sans commentaire
    \item D’accord
    \item Totalement
    d’accord
\end{enumerate}

\noindent
Le tableau d’effectifs pour les variables SEXE et réponses est présenté dans le tableau ci-dessous.

\begin{center}
    \begin{tabular}{| c | c | c | c | c | c | c |}
        \hline
        \multicolumn{2}{| c |}{\textbf{REPONSE}} & 1 & 2 & 3 & 4 & 5 \\ \hline
        \multirow{2}{*}{\textbf{GENRE}} & Homme  & 3 & 9 & 18 & 36 & 29 \\ \cline{2-7}
                                        & Femme  & 3 & 3 & 1 & 14 & 13 \\ \hline
    \end{tabular}
\end{center}
\vspace{.5cm}


%%%%%
\begin{itemize}[label={},itemindent=-2em,leftmargin=2em]
    \item \textbf{Question~4~:} Peut-on conclure que les étudiants masculins et féminins ont un comportement différent
    lors de l’"initiation" ? Quel test allez vous effectuer pour le prouver ?
\end{itemize}
\vspace{.2cm}

\noindent
Nous pouvons dans un premier temps établir le tableau avec les paramètres $n_{i}$ et $n_{j}$~:
\vspace{.1cm}

\begin{lstlisting}[style=myPython, caption=Code Python pour $n_{i}$ et $n_{j}$, frame=lines]
nij = np.array([[3, 9, 18, 36, 29],
[3, 3, 1, 14, 13]])

sum_ni = np.array([np.sum(nij[0, :]), np.sum(nij[1, :])])
sum_nj = np.array([np.sum(nij[:, 0]), np.sum(nij[:, 1]),
    np.sum(nij[:, 2]), np.sum(nij[:, 3]), np.sum(nij[:, 4])])
sum_nij = int(np.array([np.sum(sum_ni[:])]))

print("Nous avons les données suivantes :")
print(" nij:\n", nij)
print(" sum_ni:", sum_ni)
print(" sum_nj:", sum_nj)
print(" sum nij:", sum_nij, end="\n\n")
\end{lstlisting}

\begin{lstlisting}[style=myLog, caption=Résultat du code, frame=lines]
Nous avons les données suivantes :
    nij:
    [[ 3  9 18 36 29]
    [ 3  3  1 14 13]]
    sum_ni: [95 34]
    sum_nj: [ 6 12 19 50 42]
    sum nij: 129
\end{lstlisting}

\begin{center}
    \begin{tabular}{| c | c | c | c | c | c | c | c |}
        \hline
        \multicolumn{2}{| c |}{\textbf{REPONSE}} & 1 & 2 & 3 & 4 & 5 & \textbf{$n_{i}$}\\ \hline
        \multirow{2}{*}{\textbf{GENRE}} & Homme  & 3 & 9 & 18 & 36 & 29 & 95\\ \cline{2-8}
                                        & Femme  & 3 & 3 & 1 & 14 & 13 & 34\\ \hline
        \multicolumn{2}{| c |}{\textbf{$n_{j}$}} & 6 & 12 & 19 & 50 & 42 & 129\\ \hline
    \end{tabular}
\end{center}
\vspace{.5cm}


\begin{enumerate}
    \item \textbf{Grandeur d'intérêt~:} Le comportement masculin et féminin est indépendant/dépendant
    \vspace{.1cm}
   \item \textbf{Hypothèse nulle, $H0$~:} Le comportement masculin et féminin est dépendant
    \vspace{.1cm}
   \item \textbf{Hypothèse alternative, $H1$~:} Le comportement masculin et féminin est indépendant
    \vspace{.1cm}
   \item \textbf{Niveau de confiance~:} $95\%$
    \vspace{.1cm}
   \item \textbf{Test statistique~:} $\chi^{2}_{0} = \sum^{p}_{i=1} \sum^{q}_{j=1} \frac{(n_{ij} - \frac{n_{i}.n_{j}}{n})^{2}}{\frac{n_{i}.n_{j}}{n}}$ estimée par $\chi^{2}_{Obs}$ à partir de l'échantillon
    \vspace{.1cm}
   \item \textbf{Rejet de $H0$ si~:}
        \begin{itemize}
            \item Région critique: $\chi^{2}_{Obs} > \chi^{2}_{(p-1)(q-1), \alpha}$
            \item p-valeur: $p-valeur < 0.05$
        \end{itemize}

    \clearpage
    \item \textbf{Calculs~:}
    \begin{itemize}
        \item Formules utilisées:
            \begin{figure}[!h]
                \centering
                \begin{minipage}{.48\linewidth}
                    \begin{equation}
                        \chi^{2}_{Obs} = \sum^{p}_{i=1} \sum^{q}_{j=1} \frac{(n_{ij} - \frac{n_{i}.n_{j}}{n})^{2}}{\frac{n_{i}.n_{j}}{n}}
                    \end{equation}
                \end{minipage}\hfill\vline
                \begin{minipage}{.48\linewidth}
                    \begin{equation}
                        \chi^{2}_{(p-1)(q-1), \alpha} = F^{-1}_{\chi^{2}_{(p-1)(q-1)}}(\alpha)
                    \end{equation}
                \end{minipage}
            \end{figure}

            \begin{equation}
                \textit{p-valeur} = 1 - F_{\chi^{2}_{(p-1)(q-1)}}(\chi^{2}_{Obs})
            \end{equation}
        
        \item Paramètres à prendre en compte~: \\
            \hspace*{1cm}$p \to$ Le nombre d'intervalles~: 2 (Homme, Femme) \\
            \hspace*{1cm}$k \to$ Le nombre de valeurs~: 5 par intervalles

        
    \end{itemize}
        \vspace{.2cm}

\begin{lstlisting}[style=myPython, caption=Code Python question 4, frame=lines]
p = 2  
q = 5  
ic = 95
alpha = 1 - ic / 100
chi2 = stats.chi2.ppf(1-alpha, ((p-1)*(q-1)))
chi2_Obs = 0

for i in range(p):
    for j in range(q):
        chi2_Obs += ((nij[i][j] - ((sum_ni[i]*sum_nj[j])/(sum_nij)))**2)/((sum_ni[i]*sum_nj[j])/(sum_nij))

p_valeur = 1 - stats.chi2.cdf(chi2_Obs, ((p-1)*(q-1)))

print("Question 4:")
print(" Chi2:", chi2)
print(" Chi2 Obs:", chi2_Obs)
print(" p-valeur:", p_valeur)
\end{lstlisting}

\begin{lstlisting}[style=myLog, caption=Résultat du code, frame=lines]
Question 4:
    Chi2: 9.487729036781154
    Chi2 Obs: 6.621370399683419
    p-valeur: 0.1573019095765884
\end{lstlisting}


    \item \textbf{Décision~:}
        \begin{figure}[!h]
            \centering
            \begin{minipage}{.48\linewidth}
                \begin{center}
                    \begin{tabular}{| c | c |}
                        \hline
                        \multicolumn{2}{| c |}{\textbf{Critéres de rejet de H0}} \\
                        pour $\alpha$ fixé & avec \textit{p-valeur} \\ \hline
                        $\chi^{2}_{Obs} > \chi^{2}_{(p-1)(q-1), \alpha}$ & $ \textit{p-valeur} < 0.05 $\\ \hline
                    \end{tabular}
                \end{center}
            \end{minipage}\hfill\vline
            \begin{minipage}{.48\linewidth}
                \begin{equation*}
                    \left .
                    \begin{aligned}
                        \chi^{2}_{Obs} = 6.621 \\
                        \chi^{2}_{(p-1)(q-1), \alpha} = 9.487\\
                        \textit{p-valeur} = 0.15
                    \end{aligned} \qquad
                    \right\} \qquad
                    \begin{aligned} 
                        \chi^{2}_{Obs} < \chi^{2}_{(p-1)(q-1), \alpha}\\
                        \textit{p-valeur} > 0.05
                    \end{aligned}
                \end{equation*}
            \end{minipage}
        \end{figure}

        Les résultats du test statistique montrent que \textit{H0} ne peut être rejeté puisqu'aucune des conditions n'est validée. \\
        Le comportement masculin et féminin est dépendant.
\end{enumerate}

