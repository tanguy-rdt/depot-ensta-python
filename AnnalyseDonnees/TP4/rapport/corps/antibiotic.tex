\vspace{1cm}
\section{Potency of an Antibiotic}
\vspace{.2cm}

\noindent
A drug manufacturer claimed that the mean potency of one of its antibiotics was $80\%$. A random sample
of $n=100$ capsules were tested and produced a sample mean of  $\overline{Xn}=79.7\%$ with a standard deviation
of $S_{n-1}$ = $.8\%$.

\vspace{.2cm}


%%%%%
\begin{itemize}[label={},itemindent=-2em,leftmargin=2em]
    \item \textbf{Question~3~:} Do the data present sufficient evidence to refute the manufacturer’s claim? Let $\alpha$ = 0.05.
\end{itemize}
\vspace{.2cm}

\begin{enumerate}
    \item \textbf{Grandeur d'intérêt~:} La puissance moyenne des antibiotiques.
    \vspace{.1cm}
    \item \textbf{Hypothèse nulle, H0~:} $\mu = \mu_{0} = 80\%$, soit la puissance moyenne des antibiotiques est de $80\%$.
    \vspace{.1cm}
    \item \textbf{Hypothèse alternative, H1~:} $\mu \neq \mu_{0} = 80\%$, soit la puissance moyenne des antibiotiques est différente de $80\%$.
    \vspace{.1cm}
    \item \textbf{Niveau de confiance~:} $95\%$
    \vspace{.1cm}
   \item \textbf{Test statistique~:}
        Test statistique bilatéral pour la moyenne d'un grand échantillon de variance connue~:
        \clearpage

        \begin{figure}[!h]
            \centering
            \begin{minipage}{.48\linewidth}
                \begin{equation}
                    z_{0} = \frac{\overline{x_{n}} - \mu_{0}}{\frac{\sigma}{\sqrt{n}}}
                    \label{eq:test_norm_var_inconnue}
                \end{equation}
            \end{minipage}\hfill\vline
            \begin{minipage}{.48\linewidth}
                \begin{equation}
                    z_{\alpha/2} = erf^{-1}(1 - \frac{\alpha}{2})
                    \label{eq:p-valeur_test_norm_var_inconnue}
                \end{equation}
            \end{minipage}
        \end{figure}

        \begin{equation}
            \textit{p-valeur} = 2 - 2 erf(\mid z_{0} \mid) 
            \label{eq:p-valeur_test_norm_var_inconnue}
        \end{equation}

        \vspace{.2cm}

\begin{lstlisting}[style=myPython, caption=Code Python question 3, frame=lines]
n = 100
u0 = 80/100
xn = 79.7/100
ecart_type_x = .8/100
ic = 95
alpha = 1 - ic / 100
z = stats.norm.ppf(1 - (alpha / 2))
z0 = (xn - u0)/(ecart_type_x/np.sqrt(n))
p_valeur = 2 - 2 * stats.norm.cdf(np.abs(z0))

print("Question 3:")
print(" z:", round(z, 3))
print(" z0:", round(z0, 3))
print(" p-valeur:", p_valeur)
\end{lstlisting}

\begin{lstlisting}[style=myLog, caption=Résultat du code, frame=lines]
Question 3:
 z: 1.96
 z0: -3.75
 p-valeur: 0.00017683457040162942
\end{lstlisting}


    \item \textbf{Rejet de H0 ?}
        \begin{figure}[!h]
            \centering
            \begin{minipage}{.48\linewidth}
                \begin{center}
                    \begin{tabular}{| c | c |}
                        \hline
                        \multicolumn{2}{| c |}{\textbf{Critéres de rejet de H0}} \\
                        pour $\alpha$ fixé & avec \textit{p-valeur} \\ \hline
                        $z_{0} > z_{\alpha/2}$ ou $z_{0} < - z_{\alpha/2}$ & $ \textit{p-valeur} < 0.05 $\\ \hline
                    \end{tabular}
                \end{center}
            \end{minipage}\hfill\vline
            \begin{minipage}{.48\linewidth}
                \begin{equation*}
                    \left .
                    \begin{aligned}
                        z_{0} = 1.96 \\
                        z_{\alpha/2} = -3.75\\
                        \textit{p-valeur} = 0.0001
                    \end{aligned} \qquad
                    \right\} \qquad
                    \begin{aligned} 
                        z_{0} > z_{\alpha/2}\\
                        \textit{p-valeur} < 0.01
                    \end{aligned}
                \end{equation*}
            \end{minipage}
        \end{figure}

        Les résultats du test statistique sont hautement significatifs, \textit{H0} peut être rejeté puisque les deux conditions sont validées avec une $\textit{p-valeur} < 0.01$. \\
        On peut dire avec un niveau de confiance de $95\%$ que la puissance moyenne des antibiotiques est différente de $80\%$.
\end{enumerate}