\section{Fitness}
\vspace{.2cm}

\noindent
Regardless of age, about $20\%$ of American adults participate in fitness activities at least twice a week.
However, these fitness activities change as the people get older, and occasionally participants become
nonparticipants as they age. In a local survey of n = 100 adults over 40 years old, a total of 15 people
indicated that they participated in a fitness activity at least twice a week.

\vspace{.2cm}



%%%%%
\begin{itemize}[label={},itemindent=-2em,leftmargin=2em]
    \item \textbf{Question~1~:} Do these data indicate that the participation rate for adults over 40 years of age is
    significantly less than the $20\%$ figure? Calculate the p-value and use it to draw the appropriate
    conclusions.
\end{itemize}
\vspace{.2cm}

\begin{enumerate}
    \item \textbf{Grandeur d'intérêt~:} Proportion de personne de plus de 40~ans participent à du fitness.
    \vspace{.1cm}
   \item \textbf{Hypothèse nulle, H0~:} $p = p0 = 20\%$, soit la proportion de personnes de plus de 40~ans qui participent à du fitness est de $20\%$.
    \vspace{.1cm}
   \item \textbf{Hypothèse alternative, H1~:} $p < p0$, soit la proportion de personnes de plus de 40~ans qui participent à du fitness est inférieur à $20\%$.
    \vspace{.1cm}
   \item \textbf{Niveau de confiance~:} $95\%$
    \vspace{.1cm}
   \item \textbf{Test statistique~:}
        Test statistique unilatérale pour une proportion d'un grand échantillon~: 

        \begin{equation*}
            n=100 \text{ et } p_{0}=0.2 \qquad
            \left\{ \qquad
            \begin{aligned}
                n(1-p_{0}) = 80 < 5 \\
                n.p_{0}=20 < 5
            \end{aligned}
            \right .
        \end{equation*}

        \begin{figure}[!h]
            \centering
            \begin{minipage}{.48\linewidth}
                \begin{equation}
                    z_{0} = \frac{\hat{p} - p_{0}}{\sqrt{\frac{p_{0}*(1-p_{0})}{n}}}
                    \label{eq:test_norm_var_inconnue}
                \end{equation}
            \end{minipage}\hfill\vline
            \begin{minipage}{.48\linewidth}
                \begin{equation}
                    z_{\alpha} = erf^{-1}(1 - \alpha) 
                    \label{eq:p-valeur_test_norm_var_inconnue}
                \end{equation}
            \end{minipage}
        \end{figure}

        \begin{equation}
            \textit{p-valeur} = erf(t_{0}) 
            \label{eq:p-valeur_test_norm_var_inconnue}
        \end{equation}

        \vspace{.2cm}

\begin{lstlisting}[style=myPython, caption=Code Python question 1, frame=lines]
n = 100
p0 = .2
p = 15/n
ic = 95
alpha = 1 - ic / 100
z = stats.norm.ppf(1 - alpha)
z0 = (p - p0)/np.sqrt(p0*(1-p0)/n)
p_valeur = stats.norm.cdf(z0)

print("Question 1:")
print(" z:", round(z, 3))
print(" z0:", round(z0, 3))
print(" p-valeur:", p_valeur, end="\n\n")
\end{lstlisting}

\begin{lstlisting}[style=myLog, caption=Résultat du code, frame=lines]
Question 1:
 z: 1.645
 z0: -1.25
 p-valeur: 0.10564977366685518
\end{lstlisting}


    \item \textbf{Rejet de H0 ?}
        \begin{figure}[!h]
            \centering
            \begin{minipage}{.48\linewidth}
                \begin{center}
                    \begin{tabular}{| c | c |}
                        \hline
                        \multicolumn{2}{| c |}{\textbf{Critéres de rejet de H0}} \\
                        pour $\alpha$ fixé & avec \textit{p-valeur} \\ \hline
                        $z_{0} < -z_{\alpha}$ & $ \textit{p-valeur} < 0.05 $\\ \hline
                    \end{tabular}
                \end{center}
            \end{minipage}\hfill\vline
            \begin{minipage}{.48\linewidth}
                \begin{equation*}
                    \left .
                    \begin{aligned}
                        z_{0} = -1,25 \\
                        -z_{\alpha} = -1,645\\
                        \textit{p-valeur} = 0,1
                    \end{aligned} \qquad
                    \right\} \qquad
                    \begin{aligned} 
                        z_{0} > -z_{\alpha}\\
                        \textit{p-valeur} > 0.05
                    \end{aligned}
                \end{equation*}
            \end{minipage}
        \end{figure}

        Les résultats du test statistique montrent que \textit{H0} ne peut être rejeté puisqu'aucune des conditions n'est validée. \\
        On ne peut pas dire que le taux de personnes âgé de plus de 40~ans qui font du fitness est inférieur à $20\%$.
\end{enumerate}