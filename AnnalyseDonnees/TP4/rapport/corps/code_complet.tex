\vspace{1cm}
\section{Code complet}
\begin{lstlisting}[style=myPython, caption=Code Python complet TP4, frame=lines]
import numpy as np
from scipy import stats

# question 1
n = 100
p0 = .2
p = 15/n
ic = 95
alpha = 1 - ic / 100
z = stats.norm.ppf(1 - alpha)
z0 = (p - p0)/np.sqrt(p0*(1-p0)/n)
p_valeur = stats.norm.cdf(z0)

print("Question 1:")
print(" z:", round(z, 3))
print(" z0:", round(z0, 3))
print(" p-valeur:", p_valeur, end="\n\n")


# question 2
x = np.array([39, 39, 40, 33, 36, 40, 37, 41, 39, 34, 42, 41, 42, 44, 42, 42, 
              39, 42, 41, 40, 43, 43, 40, 39, 37])
n = len(x)
u0 = 37.5
xn = np.mean(x)
ecart_type_x = np.std(x, ddof=1)
ic = 95
alpha = 1 - ic / 100
t = stats.t.ppf(1 - (alpha / 2), n - 1)
t0 = (xn - u0)/(ecart_type_x/np.sqrt(n))
p_valeur = 2 - 2 * stats.norm.cdf(np.abs(t0))

print("Question 2:")
print(" t:", round(t, 3))
print(" t0:", round(t0, 3))
print(" p-valeur:", p_valeur, end="\n\n")


# question 3
n = 100
u0 = 80/100
xn = 79.7/100
ecart_type_x = .8/100
ic = 95
alpha = 1 - ic / 100
z = stats.norm.ppf(1 - (alpha / 2))
z0 = (xn - u0)/(ecart_type_x/np.sqrt(n))
p_valeur = 2 - 2 * stats.norm.cdf(np.abs(z0))

print("Question 3:")
print(" z:", round(z, 3))
print(" z0:", round(z0, 3))
print(" p-valeur:", p_valeur)
\end{lstlisting}
